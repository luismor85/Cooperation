% Options for packages loaded elsewhere
\PassOptionsToPackage{unicode}{hyperref}
\PassOptionsToPackage{hyphens}{url}
%
\documentclass[
]{article}
\usepackage{lmodern}
\usepackage{amssymb,amsmath}
\usepackage{ifxetex,ifluatex}
\ifnum 0\ifxetex 1\fi\ifluatex 1\fi=0 % if pdftex
  \usepackage[T1]{fontenc}
  \usepackage[utf8]{inputenc}
  \usepackage{textcomp} % provide euro and other symbols
\else % if luatex or xetex
  \usepackage{unicode-math}
  \defaultfontfeatures{Scale=MatchLowercase}
  \defaultfontfeatures[\rmfamily]{Ligatures=TeX,Scale=1}
\fi
% Use upquote if available, for straight quotes in verbatim environments
\IfFileExists{upquote.sty}{\usepackage{upquote}}{}
\IfFileExists{microtype.sty}{% use microtype if available
  \usepackage[]{microtype}
  \UseMicrotypeSet[protrusion]{basicmath} % disable protrusion for tt fonts
}{}
\makeatletter
\@ifundefined{KOMAClassName}{% if non-KOMA class
  \IfFileExists{parskip.sty}{%
    \usepackage{parskip}
  }{% else
    \setlength{\parindent}{0pt}
    \setlength{\parskip}{6pt plus 2pt minus 1pt}}
}{% if KOMA class
  \KOMAoptions{parskip=half}}
\makeatother
\usepackage{xcolor}
\IfFileExists{xurl.sty}{\usepackage{xurl}}{} % add URL line breaks if available
\IfFileExists{bookmark.sty}{\usepackage{bookmark}}{\usepackage{hyperref}}
\hypersetup{
  pdftitle={Cooperation in innovation},
  pdfauthor={luismor},
  hidelinks,
  pdfcreator={LaTeX via pandoc}}
\urlstyle{same} % disable monospaced font for URLs
\usepackage[margin=1in]{geometry}
\usepackage{color}
\usepackage{fancyvrb}
\newcommand{\VerbBar}{|}
\newcommand{\VERB}{\Verb[commandchars=\\\{\}]}
\DefineVerbatimEnvironment{Highlighting}{Verbatim}{commandchars=\\\{\}}
% Add ',fontsize=\small' for more characters per line
\usepackage{framed}
\definecolor{shadecolor}{RGB}{248,248,248}
\newenvironment{Shaded}{\begin{snugshade}}{\end{snugshade}}
\newcommand{\AlertTok}[1]{\textcolor[rgb]{0.94,0.16,0.16}{#1}}
\newcommand{\AnnotationTok}[1]{\textcolor[rgb]{0.56,0.35,0.01}{\textbf{\textit{#1}}}}
\newcommand{\AttributeTok}[1]{\textcolor[rgb]{0.77,0.63,0.00}{#1}}
\newcommand{\BaseNTok}[1]{\textcolor[rgb]{0.00,0.00,0.81}{#1}}
\newcommand{\BuiltInTok}[1]{#1}
\newcommand{\CharTok}[1]{\textcolor[rgb]{0.31,0.60,0.02}{#1}}
\newcommand{\CommentTok}[1]{\textcolor[rgb]{0.56,0.35,0.01}{\textit{#1}}}
\newcommand{\CommentVarTok}[1]{\textcolor[rgb]{0.56,0.35,0.01}{\textbf{\textit{#1}}}}
\newcommand{\ConstantTok}[1]{\textcolor[rgb]{0.00,0.00,0.00}{#1}}
\newcommand{\ControlFlowTok}[1]{\textcolor[rgb]{0.13,0.29,0.53}{\textbf{#1}}}
\newcommand{\DataTypeTok}[1]{\textcolor[rgb]{0.13,0.29,0.53}{#1}}
\newcommand{\DecValTok}[1]{\textcolor[rgb]{0.00,0.00,0.81}{#1}}
\newcommand{\DocumentationTok}[1]{\textcolor[rgb]{0.56,0.35,0.01}{\textbf{\textit{#1}}}}
\newcommand{\ErrorTok}[1]{\textcolor[rgb]{0.64,0.00,0.00}{\textbf{#1}}}
\newcommand{\ExtensionTok}[1]{#1}
\newcommand{\FloatTok}[1]{\textcolor[rgb]{0.00,0.00,0.81}{#1}}
\newcommand{\FunctionTok}[1]{\textcolor[rgb]{0.00,0.00,0.00}{#1}}
\newcommand{\ImportTok}[1]{#1}
\newcommand{\InformationTok}[1]{\textcolor[rgb]{0.56,0.35,0.01}{\textbf{\textit{#1}}}}
\newcommand{\KeywordTok}[1]{\textcolor[rgb]{0.13,0.29,0.53}{\textbf{#1}}}
\newcommand{\NormalTok}[1]{#1}
\newcommand{\OperatorTok}[1]{\textcolor[rgb]{0.81,0.36,0.00}{\textbf{#1}}}
\newcommand{\OtherTok}[1]{\textcolor[rgb]{0.56,0.35,0.01}{#1}}
\newcommand{\PreprocessorTok}[1]{\textcolor[rgb]{0.56,0.35,0.01}{\textit{#1}}}
\newcommand{\RegionMarkerTok}[1]{#1}
\newcommand{\SpecialCharTok}[1]{\textcolor[rgb]{0.00,0.00,0.00}{#1}}
\newcommand{\SpecialStringTok}[1]{\textcolor[rgb]{0.31,0.60,0.02}{#1}}
\newcommand{\StringTok}[1]{\textcolor[rgb]{0.31,0.60,0.02}{#1}}
\newcommand{\VariableTok}[1]{\textcolor[rgb]{0.00,0.00,0.00}{#1}}
\newcommand{\VerbatimStringTok}[1]{\textcolor[rgb]{0.31,0.60,0.02}{#1}}
\newcommand{\WarningTok}[1]{\textcolor[rgb]{0.56,0.35,0.01}{\textbf{\textit{#1}}}}
\usepackage{graphicx,grffile}
\makeatletter
\def\maxwidth{\ifdim\Gin@nat@width>\linewidth\linewidth\else\Gin@nat@width\fi}
\def\maxheight{\ifdim\Gin@nat@height>\textheight\textheight\else\Gin@nat@height\fi}
\makeatother
% Scale images if necessary, so that they will not overflow the page
% margins by default, and it is still possible to overwrite the defaults
% using explicit options in \includegraphics[width, height, ...]{}
\setkeys{Gin}{width=\maxwidth,height=\maxheight,keepaspectratio}
% Set default figure placement to htbp
\makeatletter
\def\fps@figure{htbp}
\makeatother
\setlength{\emergencystretch}{3em} % prevent overfull lines
\providecommand{\tightlist}{%
  \setlength{\itemsep}{0pt}\setlength{\parskip}{0pt}}
\setcounter{secnumdepth}{-\maxdimen} % remove section numbering

\title{Cooperation in innovation}
\author{luismor}
\date{3/25/2021}

\begin{document}
\maketitle

\hypertarget{section}{%
\subsection{}\label{section}}

\begin{Shaded}
\begin{Highlighting}[]
\CommentTok{#Carga de datos}
\KeywordTok{library}\NormalTok{(readxl)}
\KeywordTok{library}\NormalTok{(dplyr)}
\end{Highlighting}
\end{Shaded}

\begin{verbatim}
## 
## Attaching package: 'dplyr'
\end{verbatim}

\begin{verbatim}
## The following objects are masked from 'package:stats':
## 
##     filter, lag
\end{verbatim}

\begin{verbatim}
## The following objects are masked from 'package:base':
## 
##     intersect, setdiff, setequal, union
\end{verbatim}

\begin{Shaded}
\begin{Highlighting}[]
\NormalTok{df <-}\StringTok{ }\KeywordTok{read_excel}\NormalTok{(}\StringTok{"/Users/unimooc/Dropbox/2021/Directorio R/Research/Cooperation/DATOS 2004-2014.xlsx"}\NormalTok{, }\DataTypeTok{sheet =} \StringTok{"2004-2014"}\NormalTok{)}
\KeywordTok{head}\NormalTok{(df)}
\end{Highlighting}
\end{Shaded}

\begin{verbatim}
## # A tibble: 6 x 41
##   IDENT ACTIVIDAD PERIODO INTEC INTECMAN  CIFRA  GTINN  INTINN TAMANO MDOLOCAL
##   <dbl>     <dbl> <chr>   <dbl> <chr>     <dbl>  <dbl>   <dbl>  <dbl>    <dbl>
## 1     1        15 2004-2~     2 2.4      7.81e6  14232 0.00182     49        1
## 2     2        17 2004-2~     2 2.4      3.19e7 291379 0.00913    212        1
## 3     3        19 2004-2~     2 2.4      4.52e6 177232 0.0392      49        1
## 4     4        18 2004-2~     2 2.4      1.33e7 172760 0.0130      96        1
## 5     5        33 2004-2~     2 2.1      2.82e8 320506 0.00114   1129        1
## 6     6        34 2004-2~     2 2.2      1.42e8 713878 0.00504    319        1
## # ... with 31 more variables: MDONAC <dbl>, MDOUE <dbl>, INNPROD <dbl>,
## #   INNOBIEN <dbl>, INNOSERV <dbl>, INNPROC <dbl>, INNFABRI <dbl>,
## #   INNLOGIS <dbl>, INNAPOYO <dbl>, COOP <dbl>, GIO1 <dbl>, GIO2 <dbl>,
## #   GIO3 <dbl>, GIO4 <dbl>, GIO5 <dbl>, GIO6 <dbl>, GIO7 <dbl>, GIO8 <dbl>,
## #   GIO9 <dbl>, F1 <dbl>, F2 <dbl>, F3 <dbl>, F4 <dbl>, F5 <dbl>, F6 <dbl>,
## #   F7 <dbl>, F8 <dbl>, F9 <dbl>, F10 <dbl>, F11 <dbl>, TIEMPO <dbl>
\end{verbatim}

\begin{Shaded}
\begin{Highlighting}[]
\NormalTok{df <-}\StringTok{ }\KeywordTok{na.omit}\NormalTok{(df)}
\KeywordTok{anyNA}\NormalTok{(df)}
\end{Highlighting}
\end{Shaded}

\begin{verbatim}
## [1] FALSE
\end{verbatim}

\begin{Shaded}
\begin{Highlighting}[]
\NormalTok{df.gr <-}\StringTok{ }\KeywordTok{filter}\NormalTok{(df, TIEMPO }\OperatorTok{<=}\StringTok{ }\DecValTok{2007}\NormalTok{)}
\NormalTok{df.cr <-}\StringTok{ }\KeywordTok{filter}\NormalTok{(df, TIEMPO }\OperatorTok\StringTok{ }\KeywordTok{c}\NormalTok{(}\DecValTok{2008}\NormalTok{, }\DecValTok{2009}\NormalTok{, }\DecValTok{2010}\NormalTok{))}
\NormalTok{df.re <-}\StringTok{ }\KeywordTok{filter}\NormalTok{(df, TIEMPO }\OperatorTok{>=}\StringTok{ }\DecValTok{2011}\NormalTok{)}
\end{Highlighting}
\end{Shaded}

\hypertarget{chi-cuadrado}{%
\subsection{Chi Cuadrado}\label{chi-cuadrado}}

Creamos un dataframe con las variables COOP y Tiempo.

\begin{Shaded}
\begin{Highlighting}[]
\NormalTok{df.or <-}\StringTok{ }\NormalTok{df[,}\KeywordTok{c}\NormalTok{(}\StringTok{"COOP"}\NormalTok{,}\StringTok{"TIEMPO"}\NormalTok{)]}
\end{Highlighting}
\end{Shaded}

Creamos variables categóricas con la inormación de ``TIEMPO'' y ``COOP''

\begin{Shaded}
\begin{Highlighting}[]
\NormalTok{facTiemp <-}\StringTok{ }\KeywordTok{factor}\NormalTok{(df}\OperatorTok{$}\NormalTok{TIEMPO,}
                 \DataTypeTok{levels =} \KeywordTok{c}\NormalTok{(}\DecValTok{2004}\NormalTok{,}\DecValTok{2005}\NormalTok{,}\DecValTok{2006}\NormalTok{,}\DecValTok{2007}\NormalTok{,}
                            \DecValTok{2008}\NormalTok{,}\DecValTok{2009}\NormalTok{,}\DecValTok{2010}
\NormalTok{                            ,}\DecValTok{2011}\NormalTok{, }\DecValTok{2012}\NormalTok{, }\DecValTok{2013}\NormalTok{, }\DecValTok{2014}\NormalTok{),}
                 \DataTypeTok{labels =} \KeywordTok{c}\NormalTok{(}\StringTok{"Growth"}\NormalTok{, }\StringTok{"Growth"}\NormalTok{, }\StringTok{"Growth"}\NormalTok{, }\StringTok{"Growth"}\NormalTok{, }
                            \StringTok{"Crisis"}\NormalTok{, }\StringTok{"Crisis"}\NormalTok{, }\StringTok{"Crisis"}\NormalTok{, }
                            \StringTok{"Recovery"}\NormalTok{, }\StringTok{"Recovery"}\NormalTok{, }\StringTok{"Recovery"}\NormalTok{, }\StringTok{"Recovery"}\NormalTok{))}
\NormalTok{df}\OperatorTok{$}\NormalTok{facTiemp <-}\StringTok{ }\NormalTok{facTiemp}

\NormalTok{facCoop <-}\StringTok{ }\KeywordTok{factor}\NormalTok{(df}\OperatorTok{$}\NormalTok{COOP,}
                 \DataTypeTok{levels =} \KeywordTok{c}\NormalTok{(}\DecValTok{0}\NormalTok{, }\DecValTok{1}\NormalTok{),}
                 \DataTypeTok{labels =} \KeywordTok{c}\NormalTok{(}\StringTok{"NoCoop"}\NormalTok{, }\StringTok{"Coop"}\NormalTok{))}

\NormalTok{df}\OperatorTok{$}\NormalTok{facCoop <-}\StringTok{ }\NormalTok{facCoop}
\end{Highlighting}
\end{Shaded}

Creamos variables dicotómicas en función de los distintos momentos del
periodo económico

\begin{Shaded}
\begin{Highlighting}[]
\CommentTok{#Etapa de crecimiento}
\NormalTok{df}\OperatorTok{$}\NormalTok{facTiempG <-}\StringTok{ }\KeywordTok{factor}\NormalTok{(df}\OperatorTok{$}\NormalTok{TIEMPO,}
                   \DataTypeTok{levels =} \KeywordTok{c}\NormalTok{(}\DecValTok{2004}\NormalTok{,}\DecValTok{2005}\NormalTok{,}\DecValTok{2006}\NormalTok{,}\DecValTok{2007}\NormalTok{,}
                              \DecValTok{2008}\NormalTok{,}\DecValTok{2009}\NormalTok{,}\DecValTok{2010}
\NormalTok{                              ,}\DecValTok{2011}\NormalTok{, }\DecValTok{2012}\NormalTok{, }\DecValTok{2013}\NormalTok{, }\DecValTok{2014}\NormalTok{),}
                   \DataTypeTok{labels =} \KeywordTok{c}\NormalTok{(}\StringTok{"1"}\NormalTok{, }\StringTok{"1"}\NormalTok{, }\StringTok{"1"}\NormalTok{, }\StringTok{"1"}\NormalTok{, }
                              \StringTok{"0"}\NormalTok{, }\StringTok{"0"}\NormalTok{, }\StringTok{"0"}\NormalTok{, }
                              \StringTok{"0"}\NormalTok{, }\StringTok{"0"}\NormalTok{, }\StringTok{"0"}\NormalTok{, }\StringTok{"0"}\NormalTok{))}

\CommentTok{#Etapa de crisis}
\NormalTok{df}\OperatorTok{$}\NormalTok{facTiempC <-}\StringTok{ }\KeywordTok{factor}\NormalTok{(df}\OperatorTok{$}\NormalTok{TIEMPO,}
                    \DataTypeTok{levels =} \KeywordTok{c}\NormalTok{(}\DecValTok{2004}\NormalTok{,}\DecValTok{2005}\NormalTok{,}\DecValTok{2006}\NormalTok{,}\DecValTok{2007}\NormalTok{,}
                               \DecValTok{2008}\NormalTok{,}\DecValTok{2009}\NormalTok{,}\DecValTok{2010}
\NormalTok{                               ,}\DecValTok{2011}\NormalTok{, }\DecValTok{2012}\NormalTok{, }\DecValTok{2013}\NormalTok{, }\DecValTok{2014}\NormalTok{),}
                    \DataTypeTok{labels =} \KeywordTok{c}\NormalTok{(}\StringTok{"0"}\NormalTok{, }\StringTok{"0"}\NormalTok{, }\StringTok{"0"}\NormalTok{, }\StringTok{"0"}\NormalTok{, }
                               \StringTok{"1"}\NormalTok{, }\StringTok{"1"}\NormalTok{, }\StringTok{"1"}\NormalTok{, }
                               \StringTok{"0"}\NormalTok{, }\StringTok{"0"}\NormalTok{, }\StringTok{"0"}\NormalTok{, }\StringTok{"0"}\NormalTok{))}

\CommentTok{#Etapa de recuperación}
\NormalTok{df}\OperatorTok{$}\NormalTok{facTiempR <-}\StringTok{ }\KeywordTok{factor}\NormalTok{(df}\OperatorTok{$}\NormalTok{TIEMPO,}
                    \DataTypeTok{levels =} \KeywordTok{c}\NormalTok{(}\DecValTok{2004}\NormalTok{,}\DecValTok{2005}\NormalTok{,}\DecValTok{2006}\NormalTok{,}\DecValTok{2007}\NormalTok{,}
                               \DecValTok{2008}\NormalTok{,}\DecValTok{2009}\NormalTok{,}\DecValTok{2010}
\NormalTok{                               ,}\DecValTok{2011}\NormalTok{, }\DecValTok{2012}\NormalTok{, }\DecValTok{2013}\NormalTok{, }\DecValTok{2014}\NormalTok{),}
                    \DataTypeTok{labels =} \KeywordTok{c}\NormalTok{(}\StringTok{"0"}\NormalTok{, }\StringTok{"0"}\NormalTok{, }\StringTok{"0"}\NormalTok{, }\StringTok{"0"}\NormalTok{, }
                               \StringTok{"0"}\NormalTok{, }\StringTok{"0"}\NormalTok{, }\StringTok{"0"}\NormalTok{, }
                               \StringTok{"1"}\NormalTok{, }\StringTok{"1"}\NormalTok{, }\StringTok{"1"}\NormalTok{, }\StringTok{"1"}\NormalTok{))}
\end{Highlighting}
\end{Shaded}

\hypertarget{chi-cuadrado---pearson-para-la-primera-hipuxf3tesis}{%
\subsubsection{Chi cuadrado - Pearson para la primera
hipótesis}\label{chi-cuadrado---pearson-para-la-primera-hipuxf3tesis}}

H1: Companies change their perspective on cooperation to develop
innovation according to the economic cycle.

\begin{Shaded}
\begin{Highlighting}[]
\NormalTok{tablaCoop <-}\KeywordTok{table}\NormalTok{(df}\OperatorTok{$}\NormalTok{facCoop, df}\OperatorTok{$}\NormalTok{facTiemp)}
\KeywordTok{chisq.test}\NormalTok{(tablaCoop,}\DataTypeTok{correct=}\OtherTok{FALSE}\NormalTok{)}
\end{Highlighting}
\end{Shaded}

\begin{verbatim}
## 
##  Pearson's Chi-squared test
## 
## data:  tablaCoop
## X-squared = 620.47, df = 2, p-value < 2.2e-16
\end{verbatim}

\begin{Shaded}
\begin{Highlighting}[]
\NormalTok{tablaCoopG <-}\KeywordTok{table}\NormalTok{(df}\OperatorTok{$}\NormalTok{facCoop, df}\OperatorTok{$}\NormalTok{facTiempG)}
\KeywordTok{chisq.test}\NormalTok{(tablaCoopG, }\DataTypeTok{correct=}\OtherTok{FALSE}\NormalTok{)}
\end{Highlighting}
\end{Shaded}

\begin{verbatim}
## 
##  Pearson's Chi-squared test
## 
## data:  tablaCoopG
## X-squared = 339.69, df = 1, p-value < 2.2e-16
\end{verbatim}

\begin{Shaded}
\begin{Highlighting}[]
\NormalTok{tablaCoopC <-}\KeywordTok{table}\NormalTok{(df}\OperatorTok{$}\NormalTok{facCoop, df}\OperatorTok{$}\NormalTok{facTiempC)}
\KeywordTok{chisq.test}\NormalTok{(tablaCoopC, }\DataTypeTok{correct=}\OtherTok{FALSE}\NormalTok{)}
\end{Highlighting}
\end{Shaded}

\begin{verbatim}
## 
##  Pearson's Chi-squared test
## 
## data:  tablaCoopC
## X-squared = 13.291, df = 1, p-value = 0.0002667
\end{verbatim}

\begin{Shaded}
\begin{Highlighting}[]
\NormalTok{tablaCoopR <-}\KeywordTok{table}\NormalTok{(df}\OperatorTok{$}\NormalTok{facCoop, df}\OperatorTok{$}\NormalTok{facTiempR)}
\KeywordTok{chisq.test}\NormalTok{(tablaCoopR, }\DataTypeTok{correct=}\OtherTok{FALSE}\NormalTok{)}
\end{Highlighting}
\end{Shaded}

\begin{verbatim}
## 
##  Pearson's Chi-squared test
## 
## data:  tablaCoopR
## X-squared = 577.95, df = 1, p-value < 2.2e-16
\end{verbatim}

Cálculo de medias

\begin{Shaded}
\begin{Highlighting}[]
  \KeywordTok{mean}\NormalTok{(df}\OperatorTok{$}\NormalTok{COOP)}
\end{Highlighting}
\end{Shaded}

\begin{verbatim}
## [1] 0.1508814
\end{verbatim}

\begin{Shaded}
\begin{Highlighting}[]
  \KeywordTok{mean}\NormalTok{(df.gr}\OperatorTok{$}\NormalTok{COOP)}
\end{Highlighting}
\end{Shaded}

\begin{verbatim}
## [1] 0.1242933
\end{verbatim}

\begin{Shaded}
\begin{Highlighting}[]
  \KeywordTok{mean}\NormalTok{(df.cr}\OperatorTok{$}\NormalTok{COOP)    }
\end{Highlighting}
\end{Shaded}

\begin{verbatim}
## [1] 0.143812
\end{verbatim}

\begin{Shaded}
\begin{Highlighting}[]
  \KeywordTok{mean}\NormalTok{(df.re}\OperatorTok{$}\NormalTok{COOP)}
\end{Highlighting}
\end{Shaded}

\begin{verbatim}
## [1] 0.199162
\end{verbatim}

Representación gráfica

\begin{Shaded}
\begin{Highlighting}[]
\KeywordTok{mosaicplot}\NormalTok{(tablaCoop, }\DataTypeTok{ylab=}\StringTok{"Economy Cycle"}\NormalTok{, }\DataTypeTok{xlab =} \StringTok{"Cooperation"}\NormalTok{,}
           \DataTypeTok{shade =}\NormalTok{ T)}
\end{Highlighting}
\end{Shaded}

\includegraphics{Cooperation_files/figure-latex/unnamed-chunk-5-1.pdf}

\hypertarget{chi-cuadrado---pearson-para-la-segunda-hipuxf3tesis}{%
\subsubsection{Chi cuadrado - Pearson para la segunda
hipótesis}\label{chi-cuadrado---pearson-para-la-segunda-hipuxf3tesis}}

H2: Companies change their views on the competitive need for innovation
depending on the economic cycle.

\begin{Shaded}
\begin{Highlighting}[]
\NormalTok{    DOI_}\DecValTok{1}\NormalTok{ <-}\KeywordTok{table}\NormalTok{(df}\OperatorTok{$}\NormalTok{GIO1, df}\OperatorTok{$}\NormalTok{facTiemp)}
    \KeywordTok{chisq.test}\NormalTok{(DOI_}\DecValTok{1}\NormalTok{,}\DataTypeTok{correct=}\OtherTok{FALSE}\NormalTok{)}
\end{Highlighting}
\end{Shaded}

\begin{verbatim}
## 
##  Pearson's Chi-squared test
## 
## data:  DOI_1
## X-squared = 723.59, df = 6, p-value < 2.2e-16
\end{verbatim}

\begin{Shaded}
\begin{Highlighting}[]
\NormalTok{    DOI_}\DecValTok{2}\NormalTok{ <-}\KeywordTok{table}\NormalTok{(df}\OperatorTok{$}\NormalTok{GIO2, df}\OperatorTok{$}\NormalTok{facTiemp)}
    \KeywordTok{chisq.test}\NormalTok{(DOI_}\DecValTok{2}\NormalTok{,}\DataTypeTok{correct=}\OtherTok{FALSE}\NormalTok{)}
\end{Highlighting}
\end{Shaded}

\begin{verbatim}
## 
##  Pearson's Chi-squared test
## 
## data:  DOI_2
## X-squared = 1281.2, df = 6, p-value < 2.2e-16
\end{verbatim}

\begin{Shaded}
\begin{Highlighting}[]
\NormalTok{    DOI_}\DecValTok{3}\NormalTok{ <-}\KeywordTok{table}\NormalTok{(df}\OperatorTok{$}\NormalTok{GIO3, df}\OperatorTok{$}\NormalTok{facTiemp)}
    \KeywordTok{chisq.test}\NormalTok{(DOI_}\DecValTok{3}\NormalTok{,}\DataTypeTok{correct=}\OtherTok{FALSE}\NormalTok{)}
\end{Highlighting}
\end{Shaded}

\begin{verbatim}
## 
##  Pearson's Chi-squared test
## 
## data:  DOI_3
## X-squared = 905.93, df = 6, p-value < 2.2e-16
\end{verbatim}

\begin{Shaded}
\begin{Highlighting}[]
\NormalTok{    DOI_}\DecValTok{4}\NormalTok{ <-}\KeywordTok{table}\NormalTok{(df}\OperatorTok{$}\NormalTok{GIO4, df}\OperatorTok{$}\NormalTok{facTiemp)}
    \KeywordTok{chisq.test}\NormalTok{(DOI_}\DecValTok{4}\NormalTok{,}\DataTypeTok{correct=}\OtherTok{FALSE}\NormalTok{)}
\end{Highlighting}
\end{Shaded}

\begin{verbatim}
## 
##  Pearson's Chi-squared test
## 
## data:  DOI_4
## X-squared = 645.66, df = 6, p-value < 2.2e-16
\end{verbatim}

\begin{Shaded}
\begin{Highlighting}[]
\NormalTok{    DOI_}\DecValTok{5}\NormalTok{ <-}\KeywordTok{table}\NormalTok{(df}\OperatorTok{$}\NormalTok{GIO5, df}\OperatorTok{$}\NormalTok{facTiemp)}
    \KeywordTok{chisq.test}\NormalTok{(DOI_}\DecValTok{5}\NormalTok{,}\DataTypeTok{correct=}\OtherTok{FALSE}\NormalTok{)}
\end{Highlighting}
\end{Shaded}

\begin{verbatim}
## 
##  Pearson's Chi-squared test
## 
## data:  DOI_5
## X-squared = 371.83, df = 6, p-value < 2.2e-16
\end{verbatim}

\begin{Shaded}
\begin{Highlighting}[]
\NormalTok{    DOI_}\DecValTok{6}\NormalTok{ <-}\KeywordTok{table}\NormalTok{(df}\OperatorTok{$}\NormalTok{GIO6, df}\OperatorTok{$}\NormalTok{facTiemp)}
    \KeywordTok{chisq.test}\NormalTok{(DOI_}\DecValTok{6}\NormalTok{,}\DataTypeTok{correct=}\OtherTok{FALSE}\NormalTok{)}
\end{Highlighting}
\end{Shaded}

\begin{verbatim}
## 
##  Pearson's Chi-squared test
## 
## data:  DOI_6
## X-squared = 1607.7, df = 6, p-value < 2.2e-16
\end{verbatim}

\begin{Shaded}
\begin{Highlighting}[]
\NormalTok{    DOI_}\DecValTok{7}\NormalTok{ <-}\KeywordTok{table}\NormalTok{(df}\OperatorTok{$}\NormalTok{GIO7, df}\OperatorTok{$}\NormalTok{facTiemp)}
    \KeywordTok{chisq.test}\NormalTok{(DOI_}\DecValTok{7}\NormalTok{,}\DataTypeTok{correct=}\OtherTok{FALSE}\NormalTok{)}
\end{Highlighting}
\end{Shaded}

\begin{verbatim}
## 
##  Pearson's Chi-squared test
## 
## data:  DOI_7
## X-squared = 7763.1, df = 6, p-value < 2.2e-16
\end{verbatim}

\begin{Shaded}
\begin{Highlighting}[]
\NormalTok{    DOI_}\DecValTok{8}\NormalTok{ <-}\KeywordTok{table}\NormalTok{(df}\OperatorTok{$}\NormalTok{GIO8, df}\OperatorTok{$}\NormalTok{facTiemp)}
    \KeywordTok{chisq.test}\NormalTok{(DOI_}\DecValTok{8}\NormalTok{,}\DataTypeTok{correct=}\OtherTok{FALSE}\NormalTok{)}
\end{Highlighting}
\end{Shaded}

\begin{verbatim}
## 
##  Pearson's Chi-squared test
## 
## data:  DOI_8
## X-squared = 748.2, df = 6, p-value < 2.2e-16
\end{verbatim}

\begin{Shaded}
\begin{Highlighting}[]
\NormalTok{    DOI_}\DecValTok{9}\NormalTok{ <-}\KeywordTok{table}\NormalTok{(df}\OperatorTok{$}\NormalTok{GIO9, df}\OperatorTok{$}\NormalTok{facTiemp)}
    \KeywordTok{chisq.test}\NormalTok{(DOI_}\DecValTok{9}\NormalTok{,}\DataTypeTok{correct=}\OtherTok{FALSE}\NormalTok{)}
\end{Highlighting}
\end{Shaded}

\begin{verbatim}
## 
##  Pearson's Chi-squared test
## 
## data:  DOI_9
## X-squared = 201.83, df = 6, p-value < 2.2e-16
\end{verbatim}

\begin{Shaded}
\begin{Highlighting}[]
    \CommentTok{#Replicar para facTiempG, facTiempC, facTiempR}
\end{Highlighting}
\end{Shaded}

Cálculo de medias

\begin{Shaded}
\begin{Highlighting}[]
  \KeywordTok{mean}\NormalTok{(df}\OperatorTok{$}\NormalTok{GIO1)}
\end{Highlighting}
\end{Shaded}

\begin{verbatim}
## [1] 2.090599
\end{verbatim}

\begin{Shaded}
\begin{Highlighting}[]
  \KeywordTok{mean}\NormalTok{(df}\OperatorTok{$}\NormalTok{GIO2)}
\end{Highlighting}
\end{Shaded}

\begin{verbatim}
## [1] 2.287346
\end{verbatim}

\begin{Shaded}
\begin{Highlighting}[]
  \KeywordTok{mean}\NormalTok{(df}\OperatorTok{$}\NormalTok{GIO3)}
\end{Highlighting}
\end{Shaded}

\begin{verbatim}
## [1] 1.968636
\end{verbatim}

\begin{Shaded}
\begin{Highlighting}[]
  \KeywordTok{mean}\NormalTok{(df.gr}\OperatorTok{$}\NormalTok{GIO1)}
\end{Highlighting}
\end{Shaded}

\begin{verbatim}
## [1] 2.160374
\end{verbatim}

\begin{Shaded}
\begin{Highlighting}[]
  \KeywordTok{mean}\NormalTok{(df.cr}\OperatorTok{$}\NormalTok{GIO1)    }
\end{Highlighting}
\end{Shaded}

\begin{verbatim}
## [1] 2.108191
\end{verbatim}

\begin{Shaded}
\begin{Highlighting}[]
  \KeywordTok{mean}\NormalTok{(df.re}\OperatorTok{$}\NormalTok{GIO1)}
\end{Highlighting}
\end{Shaded}

\begin{verbatim}
## [1] 1.964908
\end{verbatim}

\begin{Shaded}
\begin{Highlighting}[]
  \CommentTok{#Replicar para todos los casos restantes}
\end{Highlighting}
\end{Shaded}

Representación gráfica

\begin{Shaded}
\begin{Highlighting}[]
\KeywordTok{par}\NormalTok{(}\DataTypeTok{mfrow =} \KeywordTok{c}\NormalTok{(}\DecValTok{3}\NormalTok{,}\DecValTok{3}\NormalTok{))}
   \KeywordTok{mosaicplot}\NormalTok{(DOI_}\DecValTok{1}\NormalTok{, }\DataTypeTok{ylab=}\StringTok{"Economy Cycle"}\NormalTok{, }\DataTypeTok{xlab =} \StringTok{"Likert scale"}\NormalTok{,}
               \DataTypeTok{shade =}\NormalTok{ T)}
    
    \KeywordTok{mosaicplot}\NormalTok{(DOI_}\DecValTok{2}\NormalTok{, }\DataTypeTok{ylab=}\StringTok{"Economy Cycle"}\NormalTok{, }\DataTypeTok{xlab =} \StringTok{"Likert scale"}\NormalTok{,}
               \DataTypeTok{shade =}\NormalTok{ T)}
    
    \KeywordTok{mosaicplot}\NormalTok{(DOI_}\DecValTok{3}\NormalTok{, }\DataTypeTok{ylab=}\StringTok{"Economy Cycle"}\NormalTok{, }\DataTypeTok{xlab =} \StringTok{"Likert scale"}\NormalTok{,}
               \DataTypeTok{shade =}\NormalTok{ T)    }

    \KeywordTok{mosaicplot}\NormalTok{(DOI_}\DecValTok{4}\NormalTok{, }\DataTypeTok{ylab=}\StringTok{"Economy Cycle"}\NormalTok{, }\DataTypeTok{xlab =} \StringTok{"Likert scale"}\NormalTok{,}
               \DataTypeTok{shade =}\NormalTok{ T)}
    
    \KeywordTok{mosaicplot}\NormalTok{(DOI_}\DecValTok{5}\NormalTok{, }\DataTypeTok{ylab=}\StringTok{"Economy Cycle"}\NormalTok{, }\DataTypeTok{xlab =} \StringTok{"Likert scale"}\NormalTok{,}
               \DataTypeTok{shade =}\NormalTok{ T)}
    
    \KeywordTok{mosaicplot}\NormalTok{(DOI_}\DecValTok{6}\NormalTok{, }\DataTypeTok{ylab=}\StringTok{"Economy Cycle"}\NormalTok{, }\DataTypeTok{xlab =} \StringTok{"Likert scale"}\NormalTok{,}
               \DataTypeTok{shade =}\NormalTok{ T)}
    
    \KeywordTok{mosaicplot}\NormalTok{(DOI_}\DecValTok{7}\NormalTok{, }\DataTypeTok{ylab=}\StringTok{"Economy Cycle"}\NormalTok{, }\DataTypeTok{xlab =} \StringTok{"Likert scale"}\NormalTok{,}
               \DataTypeTok{shade =}\NormalTok{ T)}
    
    \KeywordTok{mosaicplot}\NormalTok{(DOI_}\DecValTok{8}\NormalTok{, }\DataTypeTok{ylab=}\StringTok{"Economy Cycle"}\NormalTok{, }\DataTypeTok{xlab =} \StringTok{"Likert scale"}\NormalTok{,}
               \DataTypeTok{shade =}\NormalTok{ T)}
    
    \KeywordTok{mosaicplot}\NormalTok{(DOI_}\DecValTok{9}\NormalTok{, }\DataTypeTok{ylab=}\StringTok{"Economy Cycle"}\NormalTok{, }\DataTypeTok{xlab =} \StringTok{"Likert scale"}\NormalTok{,}
               \DataTypeTok{shade =}\NormalTok{ T)}
\end{Highlighting}
\end{Shaded}

\includegraphics{Cooperation_files/figure-latex/unnamed-chunk-8-1.pdf}

\hypertarget{chi-cuadrado---pearson-para-la-tercera-hipuxf3tesis}{%
\subsubsection{Chi cuadrado - Pearson para la tercera
hipótesis}\label{chi-cuadrado---pearson-para-la-tercera-hipuxf3tesis}}

H3: Companies change their views on the problems and barriers to entry
for developing innovation according to the economic cycle.

Mismo proceso que anterior.

\hypertarget{u-de-mann-whitney}{%
\subsection{U de Mann-Whitney}\label{u-de-mann-whitney}}

\hypertarget{mann-whitney-para-la-cuarta-hipuxf3tesis}{%
\subsubsection{Mann-Whitney para la cuarta
hipótesis}\label{mann-whitney-para-la-cuarta-hipuxf3tesis}}

H4: The firms' perspective on the competitive importance of innovation
and the possible problems for the development of innovation conditions
their response on whether or not to cooperate with technology centres.

\begin{Shaded}
\begin{Highlighting}[]
\KeywordTok{library}\NormalTok{(survival)}
\KeywordTok{library}\NormalTok{(coin)}
\end{Highlighting}
\end{Shaded}

P-valores de ManW

\begin{Shaded}
\begin{Highlighting}[]
\KeywordTok{wilcox.test}\NormalTok{(GIO1}\OperatorTok{~}\NormalTok{COOP,}\DataTypeTok{data=}\NormalTok{df, }\DataTypeTok{Paired =} \OtherTok{TRUE}\NormalTok{, }\DataTypeTok{exact =} \OtherTok{FALSE}\NormalTok{)}
\end{Highlighting}
\end{Shaded}

\begin{verbatim}
## 
##  Wilcoxon rank sum test with continuity correction
## 
## data:  GIO1 by COOP
## W = 522146456, p-value < 2.2e-16
## alternative hypothesis: true location shift is not equal to 0
\end{verbatim}

\begin{Shaded}
\begin{Highlighting}[]
\KeywordTok{wilcox.test}\NormalTok{(GIO2}\OperatorTok{~}\NormalTok{COOP,}\DataTypeTok{data=}\NormalTok{df, }\DataTypeTok{Paired =} \OtherTok{TRUE}\NormalTok{, }\DataTypeTok{exact =} \OtherTok{FALSE}\NormalTok{)}
\end{Highlighting}
\end{Shaded}

\begin{verbatim}
## 
##  Wilcoxon rank sum test with continuity correction
## 
## data:  GIO2 by COOP
## W = 522807048, p-value < 2.2e-16
## alternative hypothesis: true location shift is not equal to 0
\end{verbatim}

\begin{Shaded}
\begin{Highlighting}[]
\KeywordTok{wilcox.test}\NormalTok{(GIO3}\OperatorTok{~}\NormalTok{COOP,}\DataTypeTok{data=}\NormalTok{df, }\DataTypeTok{Paired =} \OtherTok{TRUE}\NormalTok{, }\DataTypeTok{exact =} \OtherTok{FALSE}\NormalTok{)}
\end{Highlighting}
\end{Shaded}

\begin{verbatim}
## 
##  Wilcoxon rank sum test with continuity correction
## 
## data:  GIO3 by COOP
## W = 513178808, p-value < 2.2e-16
## alternative hypothesis: true location shift is not equal to 0
\end{verbatim}

\begin{Shaded}
\begin{Highlighting}[]
\CommentTok{#Replicar para todas las posiblidades}
\end{Highlighting}
\end{Shaded}

Valores Z de ManW

\begin{Shaded}
\begin{Highlighting}[]
\NormalTok{df}\OperatorTok{$}\NormalTok{COOP <-}\StringTok{ }\KeywordTok{as.factor}\NormalTok{(df}\OperatorTok{$}\NormalTok{COOP)}
\NormalTok{df.gr}\OperatorTok{$}\NormalTok{COOP <-}\StringTok{ }\KeywordTok{as.factor}\NormalTok{(df.gr}\OperatorTok{$}\NormalTok{COOP)}
\NormalTok{df.cr}\OperatorTok{$}\NormalTok{COOP <-}\StringTok{ }\KeywordTok{as.factor}\NormalTok{(df.cr}\OperatorTok{$}\NormalTok{COOP)}
\NormalTok{df.re}\OperatorTok{$}\NormalTok{COOP <-}\StringTok{ }\KeywordTok{as.factor}\NormalTok{(df.re}\OperatorTok{$}\NormalTok{COOP)}

\KeywordTok{wilcox_test}\NormalTok{(df}\OperatorTok{$}\NormalTok{GIO1 }\OperatorTok{~}\StringTok{ }\NormalTok{df}\OperatorTok{$}\NormalTok{COOP)}
\end{Highlighting}
\end{Shaded}

\begin{verbatim}
## 
##  Asymptotic Wilcoxon-Mann-Whitney Test
## 
## data:  df$GIO1 by df$COOP (0, 1)
## Z = 38.661, p-value < 2.2e-16
## alternative hypothesis: true mu is not equal to 0
\end{verbatim}

\begin{Shaded}
\begin{Highlighting}[]
\KeywordTok{wilcox_test}\NormalTok{(df}\OperatorTok{$}\NormalTok{GIO2 }\OperatorTok{~}\StringTok{ }\NormalTok{df}\OperatorTok{$}\NormalTok{COOP)}
\end{Highlighting}
\end{Shaded}

\begin{verbatim}
## 
##  Asymptotic Wilcoxon-Mann-Whitney Test
## 
## data:  df$GIO2 by df$COOP (0, 1)
## Z = 38.439, p-value < 2.2e-16
## alternative hypothesis: true mu is not equal to 0
\end{verbatim}

\begin{Shaded}
\begin{Highlighting}[]
\KeywordTok{wilcox_test}\NormalTok{(df}\OperatorTok{$}\NormalTok{GIO2 }\OperatorTok{~}\StringTok{ }\NormalTok{df}\OperatorTok{$}\NormalTok{COOP)}
\end{Highlighting}
\end{Shaded}

\begin{verbatim}
## 
##  Asymptotic Wilcoxon-Mann-Whitney Test
## 
## data:  df$GIO2 by df$COOP (0, 1)
## Z = 38.439, p-value < 2.2e-16
## alternative hypothesis: true mu is not equal to 0
\end{verbatim}

\begin{Shaded}
\begin{Highlighting}[]
\CommentTok{#La función wilcox_test necesita que se le indique una variable como factor.}
\CommentTok{#Repetir para todas las posibilidades}
\end{Highlighting}
\end{Shaded}

Cálculo de medias (para las que cooperan y para las que no)

\begin{Shaded}
\begin{Highlighting}[]
\NormalTok{df}\FloatTok{.0}\NormalTok{ <-}\StringTok{ }\KeywordTok{filter}\NormalTok{(df, COOP }\OperatorTok{==}\StringTok{ "0"}\NormalTok{)}
\NormalTok{df}\FloatTok{.1}\NormalTok{ <-}\StringTok{ }\KeywordTok{filter}\NormalTok{(df, COOP }\OperatorTok{==}\StringTok{ "1"}\NormalTok{)}

\KeywordTok{mean}\NormalTok{(df}\FloatTok{.0}\OperatorTok{$}\NormalTok{GIO1)}
\end{Highlighting}
\end{Shaded}

\begin{verbatim}
## [1] 2.155942
\end{verbatim}

\begin{Shaded}
\begin{Highlighting}[]
\KeywordTok{mean}\NormalTok{(df}\FloatTok{.1}\OperatorTok{$}\NormalTok{GIO1)}
\end{Highlighting}
\end{Shaded}

\begin{verbatim}
## [1] 1.722867
\end{verbatim}

\begin{Shaded}
\begin{Highlighting}[]
\NormalTok{df.gr}\FloatTok{.0}\NormalTok{ <-}\StringTok{ }\KeywordTok{filter}\NormalTok{(df.gr, COOP }\OperatorTok{==}\StringTok{ "0"}\NormalTok{)}
\NormalTok{df.gr}\FloatTok{.1}\NormalTok{ <-}\StringTok{ }\KeywordTok{filter}\NormalTok{(df.gr, COOP }\OperatorTok{==}\StringTok{ "1"}\NormalTok{)}

\KeywordTok{mean}\NormalTok{(df.gr}\FloatTok{.0}\OperatorTok{$}\NormalTok{GIO1)}
\end{Highlighting}
\end{Shaded}

\begin{verbatim}
## [1] 2.204159
\end{verbatim}

\begin{Shaded}
\begin{Highlighting}[]
\KeywordTok{mean}\NormalTok{(df.gr}\FloatTok{.1}\OperatorTok{$}\NormalTok{GIO1)}
\end{Highlighting}
\end{Shaded}

\begin{verbatim}
## [1] 1.851886
\end{verbatim}

\begin{Shaded}
\begin{Highlighting}[]
\CommentTok{#Replicar para todos los valores necesarios}
\end{Highlighting}
\end{Shaded}

Representación gráfica

\begin{Shaded}
\begin{Highlighting}[]
\KeywordTok{par}\NormalTok{(}\DataTypeTok{mfrow =} \KeywordTok{c}\NormalTok{(}\DecValTok{3}\NormalTok{,}\DecValTok{3}\NormalTok{))}

\NormalTok{DOI_}\DecValTok{1}\NormalTok{ <-}\KeywordTok{table}\NormalTok{(df}\OperatorTok{$}\NormalTok{GIO1, df}\OperatorTok{$}\NormalTok{COOP)}
\KeywordTok{mosaicplot}\NormalTok{(DOI_}\DecValTok{1}\NormalTok{, }\DataTypeTok{ylab=}\StringTok{"Cooperation"}\NormalTok{, }\DataTypeTok{xlab =} \StringTok{"Likert scale"}\NormalTok{,}
           \DataTypeTok{shade =}\NormalTok{ T)}

\NormalTok{DOI_}\DecValTok{2}\NormalTok{ <-}\KeywordTok{table}\NormalTok{(df}\OperatorTok{$}\NormalTok{GIO2, df}\OperatorTok{$}\NormalTok{COOP)}
\KeywordTok{mosaicplot}\NormalTok{(DOI_}\DecValTok{2}\NormalTok{, }\DataTypeTok{ylab=}\StringTok{"Cooperation"}\NormalTok{, }\DataTypeTok{xlab =} \StringTok{"Likert scale"}\NormalTok{,}
           \DataTypeTok{shade =}\NormalTok{ T)}

\NormalTok{DOI_}\DecValTok{3}\NormalTok{ <-}\KeywordTok{table}\NormalTok{(df}\OperatorTok{$}\NormalTok{GIO3, df}\OperatorTok{$}\NormalTok{COOP)}
\KeywordTok{mosaicplot}\NormalTok{(DOI_}\DecValTok{3}\NormalTok{, }\DataTypeTok{ylab=}\StringTok{"Cooperation"}\NormalTok{, }\DataTypeTok{xlab =} \StringTok{"Likert scale"}\NormalTok{,}
           \DataTypeTok{shade =}\NormalTok{ T)}

\NormalTok{DOI_}\DecValTok{4}\NormalTok{ <-}\KeywordTok{table}\NormalTok{(df}\OperatorTok{$}\NormalTok{GIO1, df}\OperatorTok{$}\NormalTok{COOP)}
\KeywordTok{mosaicplot}\NormalTok{(DOI_}\DecValTok{1}\NormalTok{, }\DataTypeTok{ylab=}\StringTok{"Cooperation"}\NormalTok{, }\DataTypeTok{xlab =} \StringTok{"Likert scale"}\NormalTok{,}
           \DataTypeTok{shade =}\NormalTok{ T)}

\NormalTok{DOI_}\DecValTok{5}\NormalTok{ <-}\KeywordTok{table}\NormalTok{(df}\OperatorTok{$}\NormalTok{GIO2, df}\OperatorTok{$}\NormalTok{COOP)}
\KeywordTok{mosaicplot}\NormalTok{(DOI_}\DecValTok{2}\NormalTok{, }\DataTypeTok{ylab=}\StringTok{"Cooperation"}\NormalTok{, }\DataTypeTok{xlab =} \StringTok{"Likert scale"}\NormalTok{,}
           \DataTypeTok{shade =}\NormalTok{ T)}

\NormalTok{DOI_}\DecValTok{6}\NormalTok{ <-}\KeywordTok{table}\NormalTok{(df}\OperatorTok{$}\NormalTok{GIO3, df}\OperatorTok{$}\NormalTok{COOP)}
\KeywordTok{mosaicplot}\NormalTok{(DOI_}\DecValTok{3}\NormalTok{, }\DataTypeTok{ylab=}\StringTok{"Cooperation"}\NormalTok{, }\DataTypeTok{xlab =} \StringTok{"Likert scale"}\NormalTok{,}
           \DataTypeTok{shade =}\NormalTok{ T)}

\NormalTok{DOI_}\DecValTok{7}\NormalTok{ <-}\KeywordTok{table}\NormalTok{(df}\OperatorTok{$}\NormalTok{GIO1, df}\OperatorTok{$}\NormalTok{COOP)}
\KeywordTok{mosaicplot}\NormalTok{(DOI_}\DecValTok{1}\NormalTok{, }\DataTypeTok{ylab=}\StringTok{"Cooperation"}\NormalTok{, }\DataTypeTok{xlab =} \StringTok{"Likert scale"}\NormalTok{,}
           \DataTypeTok{shade =}\NormalTok{ T)}

\NormalTok{DOI_}\DecValTok{8}\NormalTok{ <-}\KeywordTok{table}\NormalTok{(df}\OperatorTok{$}\NormalTok{GIO2, df}\OperatorTok{$}\NormalTok{COOP)}
\KeywordTok{mosaicplot}\NormalTok{(DOI_}\DecValTok{2}\NormalTok{, }\DataTypeTok{ylab=}\StringTok{"Cooperation"}\NormalTok{, }\DataTypeTok{xlab =} \StringTok{"Likert scale"}\NormalTok{,}
           \DataTypeTok{shade =}\NormalTok{ T)}

\NormalTok{DOI_}\DecValTok{9}\NormalTok{ <-}\KeywordTok{table}\NormalTok{(df}\OperatorTok{$}\NormalTok{GIO3, df}\OperatorTok{$}\NormalTok{COOP)}
\KeywordTok{mosaicplot}\NormalTok{(DOI_}\DecValTok{3}\NormalTok{, }\DataTypeTok{ylab=}\StringTok{"Cooperation"}\NormalTok{, }\DataTypeTok{xlab =} \StringTok{"Likert scale"}\NormalTok{,}
           \DataTypeTok{shade =}\NormalTok{ T)}
\end{Highlighting}
\end{Shaded}

\includegraphics{Cooperation_files/figure-latex/unnamed-chunk-13-1.pdf}

\end{document}
